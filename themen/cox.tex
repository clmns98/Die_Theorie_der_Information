\section{Cox's Theorem}
\sectionauthor{Patricia Hackl, Mara Germann, (Lara Müller)}
%Wie berechnet man die Wahscheinlichkeit eines Ereignisses, dem ein anderes Ereignis zu Grunde liegt? Um diese Frage zu beantworten, bedarf es der Wahrscheinlichkeitsrechnung. 
Wie wird die Wahscheinlichkeit eines Ereignisses, dem ein anderes Ereignis zugrunde liegt, berechnet? Um diese Frage zu beantworten, bedarf es der Wahrscheinlichkeitsrechnung.

Bei Cox's Theorem wird die Wahrscheinlichkeitsrechnung aus der Logik hergeleitet. Im Grenzfall für absolute Sicherheit, für das Eintreten beziehungsweise Nichteintreten von Ereignissen, geht die Wahrscheinlichkeitsrechnung in wahr/falsch Aussagen und boolsche Logik über. Mit Cox's Theorem entsteht eine in sich konsistente (einheitliche) Theorie für die Wahrscheinlichkeitsrechnung.
Aus Cox's Theorem lässt sich herleiten, dass man Wahrscheinlichkeiten als Grad der Plausibilität interpretieren kann. Es wird später deutlich werden, dass die Plausibilität der Wahrscheinlichkeit entspricht.
Cox's Theorem beruht auf 3 Axiomen, die die Begründung der Bayes'schen Wahrscheinlichkeitsrechnung sind.


\begin{enumerate}
 \item Der \textbf{Grad der Plausibilität} eines Ereignisses $B$ unter der Bedingung, dass $A$ wahr ist $\{B|A\}$, wird als \textbf {reelle Zahl} dargestellt. Für hohe Plausibilitäten werden hohe Zahlenwerte gewählt. Dies ermöglicht den \textbf {universellen} Vergleich voneinander unabhängiger Plausibilitäten.
 
 \item Sinnvolle Ergebnisse werden unter qualitativem Miteinbezug des \textbf {Verstandes} und durch logische Schlussfolgerungen erzielt.
 \item Es müssen \textbf {alle }verfügbaren Informationen miteinbezogen werden. An die Schlussfolgerungen wird die Anforderung der \textbf {Konsistenz }gestellt, sodass alle Sätze, die gleiches Wissen vermitteln, auf gleiche Plausibilitäten hinführen müssen.
\end{enumerate}


Aus Cox's Theorem lassen sich unter anderem die Summen- und Produktregel, die essentiell für die Wahrscheinlichkeitsrechnung sind, aus einfachen Axiomen herleiten.
Es verbindet die Logik mit der Wahrscheinlichkeitsrechnung. Im Folgenden sind $A$, $B$ und $C$ drei Ereignisse.
\\

Die Produktregel beschreibt die Plausibilität $w$ der Ereignisse $B$ und $C$ unter der Bedingung, dass $A$ wahr ist $w(BC|A)$. Dies entspricht der Plausibilität des Eintretens von $B$ unter der Bedingung, dass $A$ bereits eingetreten ist $w(B|A)$, 
multipliziert mit der Plausibilität des Eintretens von $C$ unter der Bedingung, dass $AB$ wahr ist $w(C|AB)$. Als Formel ausgedrückt:

\begin{equation}
    w(\{BC|A\})=w(\{B|A\})\cdot w(\{C|AB\}) .
\end{equation}



\noindent Die Produktregel kann auf bedingte Wahrscheinlichkeiten übertragen werden:
\begin {displaymath}
P(B \wedge C|A) = P(B|A)\cdot P(C|AB).
\end{displaymath}
Dabei versteht man unter $P(B \wedge C|A)$ die Wahrscheinlichkeit für $B$ und $C$ unter der Bedingung $A$.
Aus der Produktregel folgt, dass 1 für die Wahrheit eines Ereignisses und 0 für die Unmöglichkeit eines Ereignisses steht. Dies dient als Grundlage für die Summenregel:
\begin{equation}
    w_{ges}=w(\{B|A\}) + w(\{\bar{B}|A\})= 1 .
\end{equation}
Die Gesamtplausibilität unter der Bedingung $A$ ist die Plausibilität von $B$ unter der Bedingung $A$ 
und die Plausibilität des Gegenereignisses von $B$ ebenfalls unter der Bedingung $A$. Die Summe der beiden Wahrscheinlichkeiten ist 1.