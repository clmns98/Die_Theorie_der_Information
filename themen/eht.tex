
\section{Radiointerferometrie-Response und Event-Horizon-Telescope-Daten}
\sectionauthor{Alexander Gitnik, Jonas Fiedler, Lara Müller, Patricia Hackl}

Zur Rekonstruktion eines Bildes des Schwarzen Lochs M87* muss basierend auf den Daten des Event-Horizon-Telescopes zunächst eine Response-Funktion implementiert werden, um die Informationen vom Signal- in den Datenraum zu transferieren $s \mapsto d$. Mit der entsprechenden Implementierung wird sich der nachfolgende Abschnitt befassen.

Es werden Daten in Form von acht csv-Dateien verwendet, die im April 2017 aufgenommen wurden. Diese stammen von mehreren Messreihen und wurden zur weiteren Verarbeitung in Form von csv-Dateien gespeichert.
Um mit den Messwerten in Python arbeiten zu können, werden die csv-Dateien in NumPy-Arrays konvertiert. Durch das NumPy-Array werden die Daten aus den csv-Dateien auf einem zweidimensionalen Gitter abgebildet.

Zunächst müssen die Frequenzen der Radiowellen ausgelesen werden. Diese elektromagnetischen Wellen werden in unmittelbarer Nähe des Schwarzen Lochs ausgesandt. Es wurden die beiden Frequenzen $f_1 = 227,0707$ GHz und $f_2 = 229,0707$ GHz betrachtet. Darüber hinaus werden auch die Verbindungsvektoren zwischen den Antennen $uvw$ ausgelesen.

Um aus den Messwerten ein Bild erstellen zu können, wird die Anzahl der Pixel sowohl in x-, als auch in y-Richtung auf zunächst $100$ festgelegt, wobei dieser Wert variabel ist. Die Größe der Pixel ergibt sich für $\delta \theta$:
\[ \delta \theta = 1.22 \cdot \displaystyle\frac{\lambda}{D} = 1.22 \cdot \displaystyle\frac{c}{f \cdot D} \]

Da es sich bei dem Event Horizon Telescope um einen Zusammenschluss von Antennen auf der ganzen Welt handelt, ergibt sich ein Durchmesser von $D_{EHT} \approx D_{Erde} \approx 10.000km$. Desweiteren beschreibt $\lambda$ die Wellenlänge der Radiowellen und $c$ deren Ausbreitungsgeschwindigkeit, die der Lichtgeschwindigkeit entspricht. In die Formel eingesetzt erhält man:

\begin{equation}
  \delta \theta = 1.22 \cdot \displaystyle\frac{3 \cdot 10^{8} \displaystyle\frac{\text{m}}{\text{s}}} {229.0707 \text{GHz} \cdot 10.000 \text{km}} = (1,598 \cdot 10^{-10}) \mu \text{as} 
\end{equation}

Damit ein erstes Bild des Schwarzen Lochs generiert werden kann, ist es notwendig, die Daten $d$ mittels folgender Formel zu beschreiben. Die Werte für $I_{Amplitude}$ und $I_{Phase}$ sind in den csv-Dateien gegeben:
\[ d = I_{Amplitude} \cdot e^{i \cdot I_{Phase}} \]

Aufgrund unvollständiger Informationen, die durch die Distanz der Antennen des Event-Horizon-Telescopes entstehen, muss zur Bilderstellung Bayes'sche Statistik angewandt werden.Im Gegensatz zu NumPy stellt NIFTy entsprechende Funktionen bereit, die dies ermöglichen. Um vom Signalraum $I$ (Signal des betrachteten Objekts) in den Datenraum $d$ (empfangene Daten) zu gelangen, nutzt man in NIFTy die Funktion \verb|dirty2vis| aus der Python-Bibliothek \verb|ducc0|. Die adjungierte Abbildung \verb|vis2dirty| ermöglicht es, aus dem bekannten Datenraum auf den Signalraum zu schließen.\\
Um diese Funktionen in NIFTy wiederholt anwenden zu können, werden sie in eine Klasse verpackt. Es handelt sich dabei um eine Klasse, die von \verb|ift.LinearOperator| - einem NIFTy-Operator - erbt. Dazu müssen unter Anderem \verb|domain| und \verb|target| initialisiert werden. Der Signalraum wird über \verb|domain| beschrieben, der Datenraum über \verb|target|. In der Klasse werden die Funktionen \verb|dirty2vis| und \verb|vis2dirty| durch die Methoden \verb|TIMES| und \verb|ADJOINT_TIMES| ersetzt.
