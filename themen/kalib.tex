\section{Kalibrierung der Radiointerferometriedaten}
\sectionauthor{Benjamin Knöbel del Olmo, Ole Fleck, Aaron Gschwendt, Mara Germann}
Die Aufgabe dieses Projektteils war es, Daten, die Teil A (Ref!) simuliert, zu korrigieren. Im Gegensatz zu Beobachtungen durch einzelne Antennenarrays wurde das Schwarze Loch M87 durch das Event Horizon Telescope vermessen. Das EHT besteht aus Arrays (Gruppen aus mehreren Radioantennen) in verschiedensten Ländern (Chile, Spanien, Hawaii etc.), was den Vorteil hat, dass der Durchmesser der Beobachtung  ansteigt und es den Forschern so ermöglicht, kleinere Winkel zu unterscheiden (Ref Radiointerferometrie wg Formel!).
Grundsätzlich möchten wir unsere Simulationsparameter so an die wirklichen Daten anpassen, dass der Unterschied zwischen beiden minimiert wird. Beim EHT kommt nun die Besonderheit der unterschiedlichen Wetterlagen an den verschiedenen Standorten ins Spiel.
Es ist mit heutigen Methoden nicht möglich, den Effekt der interferierenden Quellen ausreichend zu bestimmen. Zwei Effekte treten auf: Zum einen entsteht durch die Bremsung der Radiowelle eine Phasenverschiebung, zum anderen wird die Signalstärke reduziert.
Diese Störungen lassen sich durch die Formel
$$d_{abt\lambda}\rightarrow g_{at\lambda} \bar{g}{bt\lambda}d{abt\lambda} $$
darstellen, wobei $a$ und $b$ für ein betrachtetes Antennenpaar (also Antenne $a$ und Antenne $b$) steht und $\bar{g}{at\lambda}$,  $\bar{g}{bt\lambda}$ zwei für die jeweiligen Antennenstörung beschreibende komplexe Zahlen sind: Der reelle Teil spiegelt die Signalstärkenreduktion wieder, während der imaginäre Teil die Phasenverschiebung repräsentiert.
Um diese Effekte zu annullieren bestimmen wir die Closure Quantity, eine bereinigte Form der gegebenen Daten. Closure Phases behandeln die Phasenverschiebung der empfangenen Welle und Closure Amplitudes die Änderung der Signalstärke. Daher muss eine Funktion
\begin{equation}d^0_{t\lambda}=f(d_{abt\lambda}\forall a,b)\end{equation}
gefunden werden, sodass die Transformationen
\begin{align}
d_{abt\lambda} &\rightarrow e^{i \phi at} e^{-i\pi bt} d_{abt\lambda }d(0)\\
d_{abt\lambda}&\rightarrow|g_{at\lambda}||g_{bt\lambda}|d_{abt\lambda}|
\end{align}
konstant sind, das heißt dass, die Störung keine Rolle mehr spielen. In der ersten Funktion werden die Verschiebungen in der Phase rausgekürtzt, in der zweiten die Amplitudenverschiebung.
Für die Closure Phases lautet die gesuchte Funktion
\begin{equation}f(d_{a,b},d_{b,c},d_{a,c})=d_{a,b}+d_{b,c}-d_{a,c}\end{equation}
Unter der Anahme, dass $\psi_n = e^{i \phi_{n,t}}$ gilt, kann man zeigen, dass durch Einsetzen in die Funktion die Verschiebung der Daten herausgerechnet wird.
\begin{align}
f(\psi_a\bar{\psi_b} d_{a,b},\psi_b\bar{\psi_c}d_{b,c},\psi_a \bar{\psi_c} d_{a,c})&=\psi_1\bar{\psi_b}d_{a,b}+\psi_b\bar{\psi_c}d_{b,c}-\psi_a \bar{\psi_c} d_{a,c}\\
&=e^{i \phi_{a,t}-i\phi_{b,t}}d_{a,b}+e^{i \phi_{b,t}-i\phi_{c,t}}d_{b,c}-e^{i \phi_{a,t}-i\phi_{c,t}}d_{a,c}\\
&=d_{a,b}+d_{b,c}-d_{a,c}\\
&=f(d_{a,b},d_{b,c},d_{a,c})
\end{align}
Für die Closure Amplitudes lautet die gesuchte Funktion:
\begin{align}
f(d_{a, b},d_{a, c},d_{a, d},d_{b, c},d_{b, d},d_{c, d})\overset{!}{=}\frac{d_{a,b}\cdot d_{c,d}}{d_{a,c} \cdot d_{b,d}}
\hspace{3mm}\text{bzw.}\hspace{3mm} \frac{d_{a,d}\cdot d_{b,c}}{d_{a,c} \cdot d_{b,d}}
\end{align}
Die Antennenpaare werden in Dreier- bzw. Vierergruppen aufgeteilt. Mann kann , in der die Spalten alle möglichen Antennenpaare darstellen und jede Zeile für eine der linearunabhängigen Möglichkeiten die Antennenpaare zu verknüpfen, eine Null bedeutet, dass das Antennenpaar nicht verwendet wird. Bei den Closure Phases stehen die Zahlen in der Matrix für das Vorzeichen, bei den Closure Amplitudes für den Vorzeichen des Exponents. Bei $n$ Antennen gibt es $\binom{n-1}{2}= \frac{1}{2}(n-1)\cdot(n-2)$ linear unabhängige Reihen in der Matrix.
Daher hat die Closure Phase Matrix $\binom{4-1}{2}=3$ Reihen und sieht folgend aus.
\begin{equation}
\bordermatrix{
~ &a,b&a,c&a,d&b,c&b,d&c,d\cr
~&1&-1&0&1&0&0 \cr
~&1&0&-1&0&1&0 \cr
~&0&1&-1&0&0&1 \cr}
\end{equation}

Um am Ende ein Bild zu erzeugen, benötigen wir den Posterior des Bayes Theorem, welcher uns ermöglicht, fehlende Daten durch wahrscheinliche zu ersetzen. Um diesen zu berechnen, setzen wir ein:

\begin{equation}
P(\xi|d)= \frac {P(d|\xi)\cdot P(\xi)}{ P(d) }
\end{equation}
\begin{equation}
P(d|\xi)=  {P(d_{ph}|\xi)\cdot P(d_{amp}|\xi)}    
\end{equation}
Die Likelihood (1.12) besteht aus zwei Teilen: $P(d_{ph}|\xi)$, der Warscheinlichkeit der Closure Phases unter der Bedingung $\xi$ und $P(d_{am}| \xi)$, der Wahrscheinlichkeit der Closure Amplitudes unter der Bedingung $\xi$.
\begin{align}
 -\log {P(\xi|d)} &=
 \mathcal{H} (\xi|d) = \mathcal{H} (d_{ph}|\xi)+\mathcal{H} (d_{amp}|\xi)+\mathcal{H} (\xi) \\
&= \frac {1}{2}\cdot \Bigg[d_{ph} - f_{ph}(R(sky(\xi))) \Bigg]^\dagger N^{-1}_{ph} \Bigg[d_{ph} - f_{ph}(R(sky(\xi))) \Bigg]  
\\  & + \frac {1}{2}\cdot \Bigg[d_{amp} - f_{amp}(R(sky(\xi))) \Bigg]^\dagger N_{amp}^{-1}\Bigg[d_{ph} - f_{ph}(R(sky(\xi))) \Bigg]
\\ & +  \frac {1}{2} \cdot \xi^\dagger \xi
\end{align}
Der Logarithmus(1.14), also $H(\xi|d)$ muss minimiert werden, um den Posterior, die Wahrscheinlichkeit, dass die Simulation durch gewählte Parameter $\xi$ gut zu den gemessenen Daten passt, zu maximieren.
Der Parameter $\xi$ wird im Folgenden so optimiert, dass man eine durch $\xi$ definierte Normalverteilung eine möglichst kleine Differenz zu den errechneten Daten erreicht(s. Kullback-Leibler-Divergenz(hier kommt noch eine Referenz!)) % REF!!!
Schlussendlich erhält man folgendes Bild:
(Hier wird das Schwarze Loch Foto eingefügt)
