\documentclass[]{dsadokumentation}

% Extra packages / definitions
\usepackage{amssymb}
\usepackage{amsthm}
\usepackage{algorithm}
\usepackage[noend]{algpseudocode}
\usepackage{wrapfig}



% Bibliography
\addbibresource{kurs4.2.bib}

% Custom Commands / definitions
\setcounter{chapter}{1}
\newcommand\myacademy{Wolfsberg 2022}


\begin{document}
\kurs{Die Theorie der Information}{Wie aus Daten Bilder werden}{example-image-a}


% ======================
% Eigenwertproblem - Lara & Chuyang
\section{Eigenwertproblem}
\sectionauthor{Lara Müller, Chuyang Wang}

\subsection{Definitionen}\label{k4.2.eigen.def}

Ein Eigenvektor ist der Vektor, welcher nach Anwendung einer Matrix immer auf einem Vielfachen von sich selbst liegt. 
Formal werden die Eigenvektoren einer quadratischen Matrix $A \in \mathbb{R}^{x \times x}$ definiert als $\vec{v} \in \mathbb{R}^{x}, \vec{v} \neq \vec{0}$, mit denen $A \cdot \vec{v} = \lambda \cdot \vec{v}$ erfüllt ist. Man nennt das Skalar $\lambda$ den zu $\vec{v}$ zugehörigen Eigenwert. 

Durch Äquivalenzumformung erhält man

\begin{equation}
  \label{k4.2.eigen.def.lgs}
  \begin{aligned}
    && A \cdot \vec{v} &= \lambda \cdot \vec{v} && \\
    &\Leftrightarrow& A \cdot \vec{v} &= \lambda E \cdot \vec{v} &\text{mit } \vec{v} = E\vec{v}& \\
    &\Leftrightarrow& (A - \lambda E) \cdot \vec{v} &= 0 && \\
    &\Leftrightarrow& B\vec{x} &= 0  \quad \quad &\text{mit } B = (A - \lambda E)&  \\ 
  \end{aligned}
\end{equation}

Die \textit{Determinante} einer Matrix ist ein skalarer Wert, welche die Eigenschaft dieser Matrix beschreibt. Dieses lineare Gleichungssystem \cref{k4.2.eigen.def.lgs} hat erst dann nicht-triviale Lösungen ($\vec{x} = 0$), wenn die Determinante von $B$ gleich $0$ ist. Also gilt $\det (B) = \det (A - \lambda E) = 0$. 

Bei dem Eigenwertproblem gilt es, diese Vektoren sowie die zugehörigen Eigenwerte zu finden. 


\subsection{Das charakteristische Polynom}

Im Allgemeinen wird das charakteristische Polynom definiert als $\chi_A (\lambda) = \det(A - \lambda E)$. Aus \cref{k4.2.eigen.def} folgt, dass man die Nullstellen dieses Polynoms finden muss, um den Eigenwert zu berechnen. 

Betrachtet man nun beispielsweise das Problem in 2D. Sei $A = \begin{pmatrix}
  a & b \\
  c & d
\end{pmatrix}$, so gilt

\begin{equation}
  \label{k4.2.eigen.charac.2d}
  \begin{aligned}
    0  
    &= \chi_A (\lambda) \\
    &= \det(A - \lambda E) \\
    &= \det \begin{pmatrix}
      a - \lambda & b \\
      c & d-\lambda
    \end{pmatrix} \\
    &= (a - \lambda) \cdot (d - \lambda) - c \cdot b
  \end{aligned}
\end{equation}

Eine allgemeine Lösung für \cref{k4.2.eigen.charac.2d} kann dann mithilfe der pq-Formel berechnet werden:

\begin{equation}
  \begin{aligned}
    \lambda = \frac{a+d}{2} \pm \sqrt{\Big(\frac{(a+d)}{2}\Big)^2-ad+cb} 
  \end{aligned}
\end{equation}


\subsection{Power Iteration}\label{k4.2.eigen.powerit}

Für $2 \times 2$ Matrizen lässt sich das Eigenwertproblem relativ gut lösen, da es eine allgemeine Formel (vgl. pq-Formel / abc-Formel) für das charakteristische Polynom existiert. Ab dem 5. Grad wird es jedoch unmöglich, eine allgemeine Formel herzuleiten \parencite{k4.2.ramond}. % Abel Ruffini Theorem 
Mit dem Power-Iteration-Algorithmus versucht man, eine Annäherung an den höchsten Eigenwert zu berechnen. Der iterative Algorithmus wählt am Anfang einen willkürlichen Wert für $b_0 \in \mathbb{R}^n$. Bei jeder Iteration aktualisiert man diesen Vektor $b$ wie folgt: 

\begin{equation}
  \begin{aligned}
    b_{k+1} = \frac{Ab_k}{\left\lVert Ab_k \right\rVert }
  \end{aligned}
\end{equation}

Nach ausreichenden Iterationen kann man den größten Eigenwert $\lambda$ berechnen, indem die Gleichung $B b_{k} = \lambda b_{k}$ nach $\lambda$ auflöst wird. 


\subsection{Anwendungen}

Eigenwerte und Eigenvektoren sind wichtige Werkzeuge für viele mathematische Rechnungen und Beweise. Ein Beispiel dafür ist die Drehung durch eine Matrix: Wenn die Matrix $A$ eine Drehung um einen bestimmten Winkel beschreibt, dann ist der Eigenvektor die Drehachse, da seine Richtung durch die Drehung nicht verändert wird. 

% \ Eigenwertproblem - Lara & Chuyang
% ======================


% Bibliography
\printbibliography{}
\end{document}
