\section{Eigenwertproblem}
\sectionauthor{Lara Müller, Chuyang Wang, Benjamin Knöbel del Olmo}

\subsection{Definitionen}\label{k4.2.eigen.def}

Die Eigenvektoren einer Matrix sind diejenigen Vektoren, welche nach Anwendung dieser Matrix immer auf einem Vielfachen von sich selbst liegen.
Formal werden die Eigenvektoren einer quadratischen Matrix $A \in \mathbb{R}^{x \times x}$ definiert als $v \in \mathbb{R}^{x}, v \neq 0$, sodass $A \cdot v = \lambda \cdot v$ erfüllt ist. Man nennt das Skalar $\lambda$ den zu $v$ zugehörigen Eigenwert.

Durch Äquivalenzumformung erhält man

\begin{equation}
  \label{k4.2.eigen.def.lgs}
  \begin{aligned}
     &                 & A \cdot v               & = \lambda \cdot v   &                                 & \\
     & \Leftrightarrow & A \cdot v               & = \lambda E \cdot v & \text{mit } v = Ev              & \\
     & \Leftrightarrow & (A - \lambda E) \cdot v & = 0                 &                                 & \\
     & \Leftrightarrow & Bx                      & = 0  \quad \quad    & \text{mit } B = (A - \lambda E) & \\
  \end{aligned}
\end{equation}

Die \emph{Determinante} einer Matrix ist ein skalarer Wert, welcher die Lösbarkeit dieser Matrix beschreibt. Dieses lineare Gleichungssystem aus \cref{k4.2.eigen.def.lgs} hat genau dann nicht-triviale Lösungen ($x \neq 0$), wenn die Determinante von $B$ gleich $0$ ist. Also gilt $\det (B) = \det (A - \lambda E) = 0$.

Bei dem Eigenwertproblem gilt es, diese Vektoren sowie die zugehörigen Eigenwerte zu finden.


\subsection{Das charakteristische Polynom}

Im Allgemeinen wird das charakteristische Polynom der Matrix $A$ definiert als $\chi_A (\lambda) = \det(A - \lambda E)$. Aus \cref{k4.2.eigen.def} folgt, dass man die Nullstellen dieses Polynoms finden muss, um den Eigenwert zu berechnen.

Betrachtet man nun beispielsweise das Problem in 2D und sei $A = \begin{pmatrix}
    a & b \\
    c & d
  \end{pmatrix}$, so gilt

\begin{equation}
  \label{k4.2.eigen.charac.2d}
  \begin{aligned}
    0
     & = \chi_A (\lambda)                              \\
     & = \det(A - \lambda E)                           \\
     & = \det \begin{pmatrix}
                a - \lambda & b         \\
                c           & d-\lambda
              \end{pmatrix}                  \\
     & = (a - \lambda) \cdot (d - \lambda) - c \cdot b
  \end{aligned}
\end{equation}

Eine allgemeine Lösung für \cref{k4.2.eigen.charac.2d} kann dann mithilfe der pq-Formel berechnet werden:

\begin{equation}
  \begin{aligned}
    \lambda = \frac{a+d}{2} \pm \sqrt{\Big(\frac{(a+d)}{2}\Big)^2-ad+cb}
  \end{aligned}
\end{equation}


\subsection{Power Iteration}\label{k4.2.eigen.powerit}

Für $2 \times 2$ Matrizen lässt sich das Eigenwertproblem relativ gut lösen, da eine allgemeine Formel (vgl. pq-Formel / abc-Formel) für das charakteristische Polynom existiert. Ab dem 5. Grad ist es jedoch unmöglich, eine allgemeine Formel herzuleiten \parencite{k4.2.ramond}. % Abel Ruffini Theorem
Mit dem Power-Iteration-Algorithmus versucht man, eine Annäherung an den höchsten Eigenwert zu berechnen. Der iterative Algorithmus wählt am Anfang einen willkürlichen Wert für $b_0 \in \mathbb{R}^n$. Bei jeder Iteration aktualisiert man diesen Vektor $b$ wie folgt:

\begin{equation}
  \begin{aligned}
    b_{k+1} = \frac{Ab_k}{\left\lVert Ab_k \right\rVert }
  \end{aligned}
\end{equation}

Nach ausreichenden Iterationen kann man den größten Eigenwert $\lambda$ berechnen, indem man die Gleichung $B b_{k} = \lambda b_{k}$ nach $\lambda$ auflöst.

\subsection{Anwendungen}

Eigenwerte und Eigenvektoren sind wichtig in der Mathematik. Ein Beispiel dafür ist die Drehung durch eine Matrix: Wenn die Matrix $A$ eine Drehung um einen bestimmten Winkel beschreibt, dann ist der Eigenvektor die Drehachse, da seine Richtung durch die Drehung nicht verändert wird.
