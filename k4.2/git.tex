\section{Git}
\sectionauthor{Jonas Fiedler, Clemens Ljungh}

Mit der Versionsmanagement-Software Git kann Kooperation zwischen mehreren Programmierern ermöglicht werden. Git beschäftigt sich hauptsächlich mit dem Verwalten von Versionen des Codes und der Zusammenführung von Code Änderungen. Dafür erstellt Git einen Versionsverlauf, welcher es ermöglicht, Versionen des Codes zurückrollen um Bugs zu identifizieren und das Nachverfolgen von Code-Änderungen von wichtigen Updates zu ermöglichen. Dabei wird zwischen lokaler Dateispeicherung und Speicherung im serverseitigen Bereich unterschieden. Der Server ermöglicht den Nutzern, Code herunterzuladen, ihn zu bearbeiten und letztendlich in überarbeiteter Version wieder an den Server zu übermitteln, wo dieser dann mit anderen Nutzern ausgetauscht werden kann.

Generell ist es in Git möglich verschiedene Versionsstränge (Branches) aufzubauen. Diese Branches ermöglichen die Modifikation an verschiedenen Bereichen und ein strukturierte Entwicklung von Features. In diesen Branches können somit bspw. Bugs unabhängig gelöst werden und neue Funktionen getestet sowie implementiert werden. Sobald diese ordnungsgemäß funktionieren, werden sie auf den nächst höheren Branch, eine in der Hierarchie übergeordnete Ebene, geschoben (Merge) und somit mit dem Rest des Codes wieder vereint.

In Git gibt es verschiedene Stadien, der \enquote{Codeverarbeitung} um eine Codeversion zu speichern. Dabei beschreibt \verb|git add| den Vorgang wo die Datei von dem Working-Directory (lokal), in die Staging Area (auch lokal) verschoben wird. Der wichtigste Befehl in Git ist \verb|git commit|, dieser hält den aktuellen Zustand des gesamten Projektes fest und speichert diesen im Localrepo. Danach ermöglicht \verb|git push| das Übertragen des Codes an den Server. Der umgekehrte Schritt dazu ist \verb|git pull|. Dies lädt die aktuellste Version des Projektes vom Remoterepo in das Localrepo des Users.

Aufgrund der diversen Verwaltungsoptionen und der Möglichkeit kollaborativ in Git zu arbeiten, wird dieses Tool von vielen Software-Developern, sowie Unternehmen verwendet.
